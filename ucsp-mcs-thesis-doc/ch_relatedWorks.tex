\chapter{Related Works}
\label{ch:relatedWorks}
Our work draws on prior research in the areas of map interpretation that is focused on extracting information from maps, automatic chart interpretation focused on analyzing charts and interactive applications from chart images that enable user-interaction.


\section{Map Interpretation}
\label{sec:mapInterpretation}
Researchers have proposed various methods to perform automatic \textit{map interpretation} \citep{Walter2011} to extract information from maps and analyze their content. For instance, \citeauthor{Dhar2006}~\citep{Dhar2006} analyze scanned topographic maps to extract and recognize symbols (\eg trees, forests, rivers, cities, huts, \etc) and text contained within the map. One of the steps is to separate the image into four layers: green elements (trees, forests), red elements (streets), blue elements (rivers, lakes) and black elements (text). The map scale and range of latitude/longitude coordinates are entered by the user to locate points on the map given their geographical coordinates. Finally, the output is an \textit{e-map} that can be used as input to \ac{GIS}. \citeauthor{Pezeshk2011}~\citep{Pezeshk2011} also worked on scanned topographic maps; their purpose was to automatically extract each component of the map in separate layers and recognize the text contained. They propose an algorithm for extracting linear features to generate a layer containing map lines (streets, roads, \etc); they then use the RANSAC algorithm~\citep{Fischler1981} to improve the text preprocessing and a \ac{HMM} to recognize texts and generate a text output layer.

These previous works focus mainly on topographic maps --- \ie maps characterized by contour lines and road lines~\citep{Pezeshk2011} --- and recognizing their symbols. Our approach automatically extracts spatial information from the geographical map contained in a map visualization; this information includes the type of geographic projection used by the map and the range of latitude and longitude values in the displayed region.


\section{Automatic Chart Interpretation}
\label{sec:chartInterpretation}
A growing number of techniques focus on the ``inverse problem" of data visualization: given a visualization, recover the underlying visual encoding and its corresponding data values~\citep{Poco2017a}. Some of these approaches have focused on \emph{data extraction}. For instance, ReVision~\citep{Savva2011} classifies images by chart type and extracts data from pie and bar charts to output a relational data table. Similarly, the VIEW system~\citep{Gao2012} extracts information from raster-format charts (\eg pie, bar, and line charts). It first distinguishes graphical and textual connected-components; depending on the graphic type it applies a different approach to extract data; finally, it generates a data table with that information. \citeauthor{Al-Zaidy2016}~\citet{Al-Zaidy2016} propose a system that extracts data values from bitmap images of bar charts and generates a semantic graph using the label roles (\eg x-title, x-labels, y-title, \etc); then, the semantic graph is used to generate a summary that describes the input image.

FigureSeer~\citep{Siegel2016} is a framework that extracts information from line charts. It detects the axes to extract their labels and infers their scales through curve fitting. To perform legend analysis, it uses a random-forest classifier~\citep{randForest2001} to determine whether or not text serves as a legend label and then obtains its symbol. The analysis of the plotting area is done using \ac{SVM}~\citep{Cortes1995} and a \ac{CNN}~\citep{LeCun1998} to learn functions and avoid problems caused by the occlusion between the lines. Another application is ChartSense~\citep{Jung2017}, an interactive system for data extraction from five types of charts: line, area, radar, bar, and pie charts. Its first step is to classify chart images using a classifier based on GoogLeNet~\citep{googlenet2014}; it then extracts the data using optimized extraction algorithms for each chart type. 
% 
These approaches extract data from charts that contain discrete legends (\eg bar, pie, area, line, or radar charts). Our work is focused on the extraction of data from visual components in map visualizations that contain continuous and quantized color legends. This chart type has not been addressed so far, despite being considered in ReVision~\citep{Savva2011} during its classification step.

On the other hand, some methods have been focused on \textit{recovering visual encoding} from a chart. \citeauthor{Harper2014}~\citep{Harper2014} present a tool to decompose and redesign visualizations created with the D3 library~\citep{Bostock2011} (\eg bar charts, line charts, scatter plots, donut charts, and choropleth). This tool extracts data, marks, and visual encoding by analyzing the \ac{SVG} elements of the chart and the data bound to those elements via JavaScript.
\citeauthor{Poco2017}~\citep{Poco2017} propose a method to recover visual encodings from bitmap images of bar charts, area charts, line charts, and scatter plots; their pipeline identifies textual elements in the image, determines their role within the chart (\eg chart title, $x$-labels, $x$-title, \etc), and recovers the text content using \ac{OCR}. They also trained a \ac{CNN}~\citep{LeCun1998} for classifying 10 chart types, which achieved an average accuracy better than ReVision and ChartSense, achieving an accuracy of 96\% for classifying maps. Using this extracted information they then recover a visual encoding specification. However, their work does not include extraction of color encodings or geographic projections.

As part of this thesis, we presented a work~\citep{Poco2017a} where we proposed a technique to extract the color encoding from discrete and continuous legends of chart images, including geographic maps. We identify the colors used and the legend texts, then recover the full color mapping (\ie associating value labels with their corresponding colors). We continue our thesis work upon that approach focusing on map visualizations; thus, we had to tackle other challenges (such as identifying map projections) and develop new applications enabled by our map image analysis pipeline.


\section{Interactive Applications from Chart Images}
\label{sec:intApps}
% FROM qualification doc (translate)
The extracted information from chart images can be useful for different applications. For instance, ReVision~\citep{Savva2011} has an interface to redesign the input chart based on the relational data table extracted by its pipeline. \citeauthor{Kong2012}~\citep{Kong2012} propose to create interactive overlays that are placed above chart bitmap images using the extracted data by ReVision~\citep{Savva2011} and Datathief \citep{Tummers2006} pipelines to improve the chart reading.

\citeauthor{Kong2014}~\citep{Kong2014} developed an interactive document viewer to improve the reading experience, in this application the user can select a paragraph in a document and some components in the charts are highlighted depending on the selection; they use ReVision~\citep{Savva2011} to extract data from bar charts and a manual annotation interface to recover the original data for other chart types. Other works like ChartSense~\citep{Jung2017} and iVoLVER~\citep{Mendez2016} use semi-automatic approaches to extract data values and also present interactive annotation interfaces to correct the output data and improve the interpretation of charts.

In the same way, we propose a web-based system named iGeoMap that enables the user-interaction on bitmap images of map visualizations. iGeoMap uses the visual encoding generated by our pipeline to create interactive overlays, generate automatic captions, recolor and reproject the input map visualization.


\section{Final Considerations}
This chapter presented some recent proposals related to our thesis work. Some research works have been focused on analyzing topographic maps to extract symbols and texts. On the other hand, other works have focused on extracting data from chart images that contain discrete color legends (\eg bar charts, line charts, pie charts) to improve the chart understanding through interactive applications.

The next chapter will present some concepts needed to understand better our work, those concepts are related to the mapping of color and geographic map properties.
